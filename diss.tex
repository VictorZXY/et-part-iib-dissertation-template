% Template for a Economics Tripos Part IIB dissertation
%\documentclass[12pt,a4paper,twoside,openright]{article}  % for double-sided printing
\documentclass[12pt,a4paper]{article}
\usepackage[natbibapa]{apacite}  % APA citation style 
\usepackage[pdfborder={0 0 0}]{hyperref}  % turns references into hyperlinks
\usepackage[margin=25mm]{geometry}  % adjusts page layout
\usepackage{abstract}
\usepackage{booktabs}  % professional-quality tables
\usepackage{graphicx}  % allows inclusion of PDF, PNG and JPG images
\usepackage{newpxtext}  % uses Adobe Palatino-like fonts
\usepackage{setspace}  % adjusts line spacing

\raggedbottom    % try to avoid widows and orphans
\sloppy
\clubpenalty1000%
\widowpenalty1000%

\setstretch{1.5}    % adjust line spacing to make more readable
\renewcommand{\abstractnamefont}{\normalfont\large\bfseries}  % larger abstract header font

\begin{document}

%%%%%%%%%%%%%%%%%%%%%%%%%%%%%%%%%%%%%%%%%%%%%%%%%%%%%%%%%%%%%%%%%%%%%%%%%%%%%%%%
% Title

\pagestyle{empty}

{\noindent \large \textbf{Candidate Number: }1234X}

\vspace*{45mm}

\begin{spacing}{1.5}
\centering
\noindent \Huge \textbf{The Approved Title of Your Dissertation Spanning Two Lines}
\end{spacing}

\vspace*{15mm}

\begin{abstract}
Lorem ipsum dolor sit amet, consetetur sadipscing elitr, sed diam nonumy eirmod
tempor invidunt ut labore et dolore magna aliquyam erat, sed diam voluptua. At
vero eos et accusam et justo duo dolores et ea rebum. Stet clita kasd gubergren,
no sea takimata sanctus est Lorem ipsum dolor sit amet. Lorem ipsum dolor sit
amet, consetetur sadipscing elitr, sed diam nonumy eirmod tempor invidunt ut
labore et dolore magna aliquyam erat, sed diam voluptua. At vero eos et accusam
et justo duo dolores et ea rebum. Stet clita kasd gubergren, no sea takimata
sanctus est Lorem ipsum dolor sit amet.
\end{abstract}

\vfill

{\noindent \large \textbf{Word Count: }XXXX}

\clearpage              % remove page number on title page
\pagenumbering{arabic}  % ... and reset page number to 1

%%%%%%%%%%%%%%%%%%%%%%%%%%%%%%%%%%%%%%%%%%%%%%%%%%%%%%%%%%%%%%%%%%%%%%%%%%%%%%%%
% Table of contents list of figures and list of tables (optional)

%\pagestyle{plain}
%
%\tableofcontents
%
%\listoffigures
%
%\listoftables

%%%%%%%%%%%%%%%%%%%%%%%%%%%%%%%%%%%%%%%%%%%%%%%%%%%%%%%%%%%%%%%%%%%%%%%%%%%%%%%%
% Now for the main sections

\pagestyle{headings}

\section{Introduction}

\subsection{Overview of the files}

This document consists of the following files:

\begin{itemize}
    \item \texttt{makefile} --- The makefile for the dissertation
    \item \texttt{diss.tex} --- The dissertation
    \item \texttt{figs} --- A directory containing diagrams and pictures
    \item \texttt{refs.bib} --- The bibliography database
\end{itemize}

\subsection{Building the document}

This document was produced using \LaTeXe~which is based upon \LaTeX~\citep{Lamport86}. 
To build the document you first need to generate \texttt{diss.aux} which, amongst
other things, contains the references used. This is done by executing the command:

\texttt{pdflatex diss}

\noindent
Then the bibliography can be generated from \texttt{refs.bib} using:

\texttt{bibtex diss}

\noindent
Finally, to ensure all the page numbering is correct run \texttt{pdflatex} on
\texttt{diss.tex} until the \texttt{.aux} files do not change. This usually
takes 2 more runs.

\subsubsection{The makefile}

To simplify the calls to \texttt{pdflatex} and \texttt{bibtex}, a makefile has
been provided. It provides the following facilities:

\begin{description}
    \item\texttt{make} \\
    Display help information.
    
    \item\texttt{make diss.pdf} \\
    Format the dissertation document as a PDF.
    
    \item\texttt{make all} \\
    Alias for \,\texttt{make diss.pdf} \,(format the dissertation document as a PDF).
    
    \item\texttt{make count} \\
    Display an estimate of the word count.
    
    \item\texttt{make clean} \\ 
    Delete all intermediate files except the source files and the resulting PDFs. 
    All these deleted files can be reconstructed by typing \,\texttt{make all}.
    
    \item\texttt{make distclean} \\ 
    Delete all files, including the resulting PDFs, except the source files. 
    All these deleted files can be reconstructed by typing \,\texttt{make all}.
\end{description}

\subsection{Counting words}

An approximate word count of the body of the dissertation may be obtained using:

\texttt{wc diss.tex}

\noindent
Alternatively, try something like:

\verb/detex diss.tex | tr -cd '0-9A-Z a-z\n' | wc -w/

\subsection{Figures}

If you wish to include a plot or other image, you can render it as shown for
Figure~\ref{fig:sample-fig}. You may use colour figures. However, it is best for
the figure captions and the paper body to be legible if the paper is printed in
either black/white or in colour.

\begin{figure}[h]
    \centering
    \includegraphics[width=.45\textwidth]{figs/sample}
    \caption{Sample figure caption}
    \label{fig:sample-fig}
\end{figure}

Simple diagrams can be written directly in \LaTeX.  For example, see
Figure~\ref{fig:latexpic1} on page~\pageref{fig:latexpic1} and see
Figure~\ref{fig:latexpic2} on page~\pageref{fig:latexpic2}.

\begin{figure}
    \centering
    \setlength{\unitlength}{1mm}
    \begin{picture}(125,100)
        \put(0,80){\framebox(50,10){AAA}}
        \put(0,60){\framebox(50,10){BBB}}
        \put(0,40){\framebox(50,10){CCC}}
        \put(0,20){\framebox(50,10){DDD}}
        \put(0,00){\framebox(50,10){EEE}}
        
        \put(75,80){\framebox(50,10){XXX}}
        \put(75,60){\framebox(50,10){YYY}}
        \put(75,40){\framebox(50,10){ZZZ}}
        
        \put(25,80){\vector(0,-1){10}}
        \put(25,60){\vector(0,-1){10}}
        \put(25,50){\vector(0,1){10}}
        \put(25,40){\vector(0,-1){10}}
        \put(25,20){\vector(0,-1){10}}
        
        \put(100,80){\vector(0,-1){10}}
        \put(100,70){\vector(0,1){10}}
        \put(100,60){\vector(0,-1){10}}
        \put(100,50){\vector(0,1){10}}
        
        \put(50,65){\vector(1,0){25}}
        \put(75,65){\vector(-1,0){25}}
    \end{picture}
    \caption{A picture composed of boxes and vectors}
    \label{fig:latexpic1}
\end{figure}

\begin{figure}
    \centering
    \setlength{\unitlength}{1mm}
    \begin{picture}(100,70)
        \put(47,65){\circle{10}}
        \put(44,64){abc}
        
        \put(37,45){\circle{10}}
        \put(37,51){\line(1,1){8.5}}
        \put(34,44){def}
        
        \put(57,25){\circle{10}}
        \put(57,31){\line(-1,3){9.5}}
        \put(57,31){\line(-3,2){15}}
        \put(54,24){ghi}
        
        \put(32,0){\framebox(10,10){A}}
        \put(52,0){\framebox(10,10){B}}
        \put(37,11){\line(0,1){28}}
        \put(37,11){\line(2,1){17}}
        \put(57,11){\line(0,2){8}}
    \end{picture}
    \caption{A diagram composed of circles, lines and boxes}
    \label{fig:latexpic2}
\end{figure}

\subsection{Tables}

Here is a simple example of a table. Note that publication-quality tables
\textit{do not contain vertical rules}. I strongly suggest the use of the
\texttt{booktabs} package, which allows for typesetting high-quality,
professional tables:

\url{https://www.ctan.org/pkg/booktabs}

\noindent
This package was used to typeset Table~\ref{table:sample-table}. There is another
example table (Table~\ref{citation-guide}) in Section~\ref{sec:citations}.

\begin{table}[h]
    \centering
    \begin{tabular}{lcr}
        \toprule
        \multicolumn{2}{c}{A Multi-Column Cell} &  \\
        \cmidrule(r){1-2}
        Left Justified & Centred & Right Justified \\
        \midrule
        First          & A       & XXX             \\
        Second         & AA      & XX              \\
        Last           & AAA     & X               \\
        \bottomrule
    \end{tabular}
    \caption{Sample table caption}
    \label{table:sample-table}
\end{table}

\subsection{Footnotes}

Here is a simple example\footnote{A footnote.} of a footnote. Footnotes should
be used sparingly. Note that footnotes are properly typeset \textit{after}
punctuation marks.\footnote{As in this example.} Footnotes \textit{are} included
in the word count.

\subsection{Citations} \label{sec:citations}

This template uses the APA citation format. Citations within the text appear in 
parentheses as~\citep{Chang14} or, if the author's name appears in the text
itself, as \citet{Chang14}. Append lowercase letters to the year in cases of
ambiguities. Treat double authors as in~\citep{Mankiw14}, but write as
in~\shortcites{Floud14}\citep{Floud14} when more than two authors are
involved (the \verb|\shortcites{key}| command is required only for the first
time you cite \texttt{key}). Collapse multiple citations as
in~\citep{Mankiw14,Floud14}. Refrain from using full citations as sentence
constituents. Instead of ``\citep{Chang14} showed that ...'', write
``\citet{Chang14} showed that ...''. Your reference list is \textit{not}
included in the word count.

Table~\ref{citation-guide} shows the supported citation syntax. You are
encouraged to use the \texttt{natbib} styles. You can use the command
\verb|\citep| (cite in parentheses) to get ``(author, year)'' citations as in
\citep{Chang14}. You can use the command \verb|\citet| (cite in text) to get
``author (year)'' citations as in \citet{Chang14}. You can use the command
\verb|\citealp| (alternative cite without parentheses) to get ``author, year''
citations (which is useful for using citations within parentheses, as in
\citealp{Chang14}).

\begin{table}[h]
    \centering
    \begin{tabular}{lll}
        \toprule
        \textbf{Command} & \textbf{Output} & \textbf{Description}\\
        \midrule
        \verb|\citep| & \citep{Chang14} & Parenthetical citation \\
        \verb|\citet| & \citet{Chang14} & Text citation \\
        \verb|\citealp| & \citealp{Chang14} & Parenthetical citation without parentheses \\
        \verb|\citealt| & \citealt{Chang14} & Text citation without parentheses \\
        \verb|\citeauthor| & \citeauthor{Chang14} & Author only (text citation) \\
        \verb|\citeyear| & \citeyear{Chang14} & Year only (no parentheses) \\
        \verb|\citeyearpar| & \citeyearpar{Chang14} & Year citation (with parentheses) \\
        \bottomrule
    \end{tabular}
    \caption{The \texttt{natbib} citation commands}
    \label{citation-guide}
\end{table}

\subsection{Further information}

See the \LaTeX~notes at

\url{https://www.cl.cam.ac.uk/teaching/current-1/TeX+MATLAB/materials.html}

\noindent
See also the Unix Tools notes at

\url{https://www.cl.cam.ac.uk/teaching/current-1/UnixTools/materials.html}

\subsection{Acknowledgements}

This template is built for my beloved girlfriend, Jing Zeng, for her Economics
Tripos Part~IIB dissertation in 2022. It is based on the Computer Science Tripos
Part~II dissertation template by Martin Richards \citeyearpar{Richards15}, which
was built upon an earlier version by Simon Moore \citeyearpar{Moore95}.

%%%%%%%%%%%%%%%%%%%%%%%%%%%%%%%%%%%%%%%%%%%%%%%%%%%%%%%%%%%%%%%%%%%%%%%%%%%%%%%%

\section{Literature Review}


%%%%%%%%%%%%%%%%%%%%%%%%%%%%%%%%%%%%%%%%%%%%%%%%%%%%%%%%%%%%%%%%%%%%%%%%%%%%%%%%

\section{Theoretical Framework}


%%%%%%%%%%%%%%%%%%%%%%%%%%%%%%%%%%%%%%%%%%%%%%%%%%%%%%%%%%%%%%%%%%%%%%%%%%%%%%%%

\section{Data}


%%%%%%%%%%%%%%%%%%%%%%%%%%%%%%%%%%%%%%%%%%%%%%%%%%%%%%%%%%%%%%%%%%%%%%%%%%%%%%%%

\section{Methodology}


%%%%%%%%%%%%%%%%%%%%%%%%%%%%%%%%%%%%%%%%%%%%%%%%%%%%%%%%%%%%%%%%%%%%%%%%%%%%%%%%

\section{Results}


%%%%%%%%%%%%%%%%%%%%%%%%%%%%%%%%%%%%%%%%%%%%%%%%%%%%%%%%%%%%%%%%%%%%%%%%%%%%%%%%

\section{Conclusion}

I hope that this rough guide to writing a dissertation is \LaTeX\ has been
helpful and saved you time.

%%%%%%%%%%%%%%%%%%%%%%%%%%%%%%%%%%%%%%%%%%%%%%%%%%%%%%%%%%%%%%%%%%%%%%%%%%%%%%%%
% The bibliography

\newpage
%\addcontentsline{toc}{section}{References}
\bibliographystyle{apacite}
\bibliography{refs}

%%%%%%%%%%%%%%%%%%%%%%%%%%%%%%%%%%%%%%%%%%%%%%%%%%%%%%%%%%%%%%%%%%%%%%%%%%%%%%%%
% The appendices

\newpage
\appendix

\section{Appendices}

Appendices \textit{are} included in the word count.

\end{document}
